\documentclass[]{article}

\usepackage{listings}
\usepackage{fullpage}
\usepackage{dirtree}
\usepackage{hyperref}
\usepackage{booktabs}
\usepackage{url}
\usepackage[utf8]{inputenc}
\usepackage{tikz}
\usepackage{graphicx}
\usepackage{mathtools}
\usepackage{float}
\usepackage{amsthm}
\usepackage{amsmath}
\usepackage{amsfonts}
\usepackage{bbm}
\usepackage[scientific-notation=true]{siunitx}

\usepackage{amssymb}
\usepackage{tabulary}

\usepackage{etoolbox}

\usepackage{algorithm}
\usepackage{algorithmic}

\title{Comments on the revised version}

\begin{document}

\maketitle

Let us first sincerely thank the reviewers for their relevant and in-depth
remarks. Their comments helped us to clarify the focus of this work, which
lead to restructure the evaluation procedure to some extent. As part of these
modifications, we changed the title to: "Online Tuning of EASY-Backfilling
using Queue Reordering Policies". It better reflects the main goal, which is
not to advocate bandit strategies, but rather to study their feasibility, and
to show where the tradeoff is compared to using simulations for tuning the
algorithms. We hope this revised version may now be considered for publication. 
We detail our answers to rewiewer's comments (bold).

\section{Reviewer 1}

\textbf{
  I basically liked this paper and support its publication.  I have only very few and minor comments.}

\textbf{In sect 1.1 you claim 2 "new" online approaches, and in 1.2 you call them "original". Your work is valid even if you
admit that both have been suggested before.
}

\textbf{Last paragraph of 1.2 is the third repetition of this.  It is redundant, despite the fact that some reviewer didn't get
it.
}

\textbf{In sect 5.1 you explain that you set lambda=1 because due to the resampling the distributions are stationary.  That is
correct.  But it would have been good to also conduct simulations on the original traces, which are probably not
stationary, and therefore there is more to gain from switching between policies.
Also, instead of starting the simulation of each period from an empty system, why not start from the situation at the
end of the previous period?
}

\textbf{Typo at end of 6.1.3: sentence with [24] appears twice.
}

%%%%%%%%%%%%%%%%%%%%%%%%%%%%%%%%%%%%%%

\section{Reviewer 2}

\textbf{In comparison to the first version, the authors have made several small changes that results in minor improvements of
the manuscript. Nevertheless, the main problems of the manuscript remain. Therefore, the position of the reviewer
remains unchanged.
}

\textbf{The authors present a study based on a selection of several strategies and parameters. They execute several simulations
and show that using a feedback mechanism improvements over the standard approach without feedback are possible. This
results is not surprising considering previous results some of which are referenced within the manuscript. It must be
noted that the authors do not suggest a specific new scheduling algorithm and show that this algorithm results in
significant improvements over existing methods.
}

\textbf{The authors mention "reducing uncertainty" in Section 1.1. It is not clear why this reduction is relevant within the
manuscript (except as a justification for adding noise?)
The first paragraph of Section 1.1 is relevant to the content of the manuscript. The study of the manuscript uses old
traces that are far from extreme scale parallel computing. The relevance to distributed computing is not clear since
only traces from parallel computers are used.
}

\textbf{Why do the authors discuss conservative backfilling in Section 2 although it does not matter in the rest of the
manuscript?
What is the meaning of "dependency of scheduling metrics on the workload"? The authors only apply a single scheduling
metric (average waiting time) and do not discuss such dependency. Why is it important that previous work does not
address this dependency in the context of this manuscript?
}

\textbf{The authors mention duality between cumulative and maximal scheduling costs. Why is this duality relevant in the
context of this manuscript? In Section 3.3 the authors claim that "Section 4.2 will outline our approach to address the
biobjective aspect of this problem". However, Section 4.2. is very brief and does not discuss such aspect. The reviewer
did not find any discussion in the rest of the manuscript. Altogether not all references cited in the manuscript are
relevant for the content of the manuscript.
}

\textbf{The design of the study is contradictory: on the one hand, the authors only select a single metric since this metric
"is one of the more commonly used objectives". On the other hand, they select 12 reordering strategies without giving
any justification based on the strategy. They simply state that they want the search space "being as semantically
diverse as possible" without stating that they achieved this goal with the selection of these strategies. Why do they
need all the 12 strategies and why not using more strategies? Again they only remark that the strategies "include most
policies from related works that we are aware of." Such justification is hardly sufficient for a scientific study.
}

\textbf{In Section 5, the authors claim that "the distribution of the job submission process does not radically differ between
consecutive periods." This claim is not true for every period length. For instance, when using 12 hours as period
length, two consecutive periods may cover day and night with different submission processes.
}

\textbf{The allocation of jobs to periods based on the submission time may be problematic. For instance a long running job may
be submitted in one period, waiting for execution in the next period and then executing in the next several periods.
Therefore, it influences several periods. The authors briefly mention such dependencies (Section 5.3) and state that
there may be a bias without discussing this issue.
}

\textbf{The authors still keep the decaying discount factor $\lambda$ without using it in the study. In an exploratory study, the
authors can either ignore such factor because it is not relevant (and saying why this claim holds) or they must include
it into the study.
}

\textbf{The authors introduce a noise factor to deal with imprecision. This way they introduce an approach to model
imprecision. But they do not show that their model is suitable in reality.
}

\textbf{In Section 5.2, the authors claim that the length of the period has little impact here as the cost metric corresponds
to the cumulative simulated waiting time for each policy. As already mentioned, this may not be true for periods that
are small compared to the waiting time of jobs.
}

\textbf{Figure 3 does not illustrate the process of the $\epsilon$-greedy strategy since there is no $\epsilon$ in Figure 3.
}

\textbf{Section 6.1.1 mentions cores in traces although several of the old traces use machines that do not have multicore
processors.
}

\textbf{The authors use 40 hours as threshold and justify this selection by referring to another publication. The reader does
not know about the arguments without reading this other publication. Since the design of the study is at the core of
this manuscript, such justification if not acceptable.
}

\textbf{The authors state that they have simulated a total of 92400 years. However, this absolute figure is not relevant
without showing the coverage of the problem space.
}

\textbf{There are only few language problems in the manuscript:
Software and work are used in the singular form. The authors must adjust the verb correspondingly (Section 1.1)
wokload in Fig. 2 and 3
}


%%%%%%%%%%%%%%%%%%%%%%%%%%%%%%%%%%%%%%%%%%%%%%%%%%%%

\section{Reviewer 3}

\textbf{I think that the simulation model presented in this paper is carefully created and it shows convincing results.
However, there are two questions about the discussion for the simulation results:
}

\textbf{The results in Table 3 and Table 4 show that the performance of the static strategy, LQF, is comparable with the
presented online tuning strategies, Full and Bandit. Which strategy is better in the real world? We can say that LQF
may be preferable because they are less complicated and run with smaller scheduling overhead. The authors should add
the discussion to compare LQF and the online tuning strategies.
}

\textbf{The authors conclude that simulation-based strategy is best but the bandit-based strategy is easier to use and cheaper
to run. How easy and how cheaper can we run the bandit-based strategy? The authors should add the quantitative
discussion how the bandit-based strategy is easy to use and saves cost (or scheduling overhead) compared with the
simulation-based strategies.
}

\bibliographystyle{abbrv}
\bibliography{bibliography}

\end{document}
